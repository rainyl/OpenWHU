\documentclass[UTF8]{ctexart}
\usepackage{graphicx}
\usepackage{ctex}
\usepackage{tikz}
\usepackage{amsmath}
\title{电动力学-第五次作业}
\author{吴远清-2018300001031}

\begin{document}
	\maketitle
    Problem 4.10\\
    Answer:\\
    (a)\\
    $$\sigma_b = P\cdot \hat{n} = kR\eqno(1.1)$$
    $$\rho_b = -\nabla \cdot P = - \frac{1}{r^3}\frac{\partial}{\partial r}(r^2kr) = -3k\eqno(1.2)$$
    (b)\\
    For $r<R$:
    $$E = \frac{1}{3\epsilon_0}\rho r \hat{r} = -\frac{k}{\epsilon_0}r\eqno(1.3)$$
    For $r>R$, we can treat it as all charge at center:
    $$Q_t = (kR)(4\pi R^2) + (-3k)(\frac{4}{3}\pi R^3)=0 \eqno(1.4)$$
    Then:
    $$E = 0\eqno(1.5)$$
    Problem 4.18\\
    Answer:\\
    (a):\\
    For the surface, we have:
    $$\int D \cdot da = Q \eqno(2.1)$$
    Then:
    $$DA = \sigma A \eqno(2.2)$$
    Which means:
    $$D = \sigma \eqno(2.3)$$
    (b)\\
    $$D = \epsilon E$$
    Combine with (2.3), we can determine the E:
    $$E = \frac{\sigma}{\epsilon_1}, in\,slab\,1 \eqno(2.4)$$
    $$E = \frac{\sigma}{\epsilon 2}, in\,slab\,2 \eqno(2.5)$$
    And in total, we have:
    $$\epsilon = \epsilon_0 \epsilon_r \eqno(2.6)$$
    So:
    $$E_1 = \frac{\sigma}{2\epsilon_0} \eqno(2.7)$$
    $$E_2 = \frac{2\sigma}{3\epsilon_0} \eqno(2.8)$$
    (c)\\
    $$P = \epsilon_0 \chi_e E \eqno(2.9)$$
    So:
    $$P = \frac{\epsilon_0\chi_e d}{\epsilon_0 \epsilon_r} = \frac{\chi_e}{\epsilon_r}\sigma \eqno(2.10)$$
    We have:
    $$\chi_e = \epsilon_r - 1 \eqno(2.11)$$
    Then:
    $$P_1 = \frac{\sigma}{2} \eqno(2.12)$$
    $$P_2 = \frac{\sigma}{3} \eqno(2.13)$$
    (d)\\
    $$V = E_1 a + E_2 a = \frac{\sigma a}{6\epsilon_0}\times(3+4) = \frac{7\sigma a}{6 \epsilon_0} \eqno(2.14)$$
    (e)\\
    $$\rho_b = 0$$
    Then:
    \begin{equation*}
        \left\{
            \begin{aligned}
                &\sigma_b = \frac{\sigma}{2} \qquad bottom\, of\, slab\, 1\\
                &\sigma_b = -\frac{\sigma}{2} \qquad top\, of\, slab\, 1\\
                &\sigma_b = \frac{\sigma}{3} \qquad bottom\, of\, slab\, 2\\
                &\sigma_b = -\frac{\sigma}{3} \qquad top\, of\, slab\, 2
            \end{aligned}
        \right.\eqno(2.15)
    \end{equation*}
    (f)\\
    For slab 1, surface charge above is:$\frac{\sigma}{2}$, surface charge below is:$-\frac{\sigma}{2}$
    $$E = \frac{\sigma}{2\epsilon_0}\eqno(2.16)$$
    For slab 2, surface charge above is:$\frac{\sigma}{3}$, surface charge below is:$-\frac{\sigma}{3}$
    $$E = \frac{2\sigma}{3\epsilon_0}\eqno(2.17)$$
    This result is same as the result in (b)\\
    Problem 4.22\\
    Answer\\
    We have the boundary condition:
    \begin{equation*}
        \left\{
            \begin{aligned}
                &V_{in} = V_{out},\, where\,s=a\\
                &\epsilon_0\frac{\partial V_{in}}{\partial s} = \epsilon_0\frac{\partial V_{out}}{\partial s}\qquad where\,s=a\\
                &\lim_{\frac{s}{a} \to \infty}V_{out} = -E_0 s\,cos\phi
            \end{aligned}
        \right.\eqno(3.1)        
    \end{equation*}
    From Problem 3.24:
    $$V_{in}(s,\phi) = \sum^{\infty}_{k=1}s^k(a_k cos\,k\phi + b_k sin \, k\phi)\eqno(3.2)$$
    And
    $$V_{out}(s,\phi) = -E_0s\,cos\phi + \sum_{k=1}^{\infty}s^{-k} (c_k cos\,k\phi + d_k sin\,k\phi) \eqno(3.3)$$
    From (3.1):
    $$\sum^{\infty}_{k=1}a^k(a_k cos\,k\phi + b_k sin \, k\phi) = -E_0s\,cos\phi + \sum_{k=1}^{\infty}a^{-k} (c_k cos\,k\phi + d_k sin\,k\phi) \eqno(3.4)$$
    $$\epsilon_r \sum^{\infty}_{k=1}ka^{k-1}(a_k cos\,k\phi + b_k sin \, k\phi) = -E_0 cos\phi - \sum_{k=1}^{\infty}ka^{-k-1}(c_k cos\,k\phi + d_k sin\,k\phi) \eqno(3.5)$$
    So:\
    $$b_k = d_k = 0\eqno(3.6)$$
    $$a_k = c_k = 0\qquad if\,k\neq 1 \eqno(3.8)$$
    And when $k=1$:
    $$a_1 = -\frac{E_0}{(1+\frac{\chi_e}{2})} \eqno(3.8)$$
    So:
    $$V_{in}(s,\phi) = -\frac{E_0}{1+\frac{\chi_e}{2}}s\,cos\phi \eqno(3.9)$$
    Problem 4.26\\
    Answer:\\
    From Example 4.5:
    \begin{equation*}
        D = 
        \left\{
            \begin{aligned}
                &0,&(r<a)\\
                &\frac{Q}{4\pi r^2}\hat{r},&(r>a)
            \end{aligned}
        \right.\eqno(4.1)
    \end{equation*}
    \begin{equation*}
        E = 
        \left\{
            \begin{aligned}
                &0,&(r<a)\\
                &\frac{Q}{4\pi \epsilon r^2}\hat{r},&(a<r<b)\\
                &\frac{Q}{4\pi \epsilon_0 r^2}\hat{r},&(r>b)
            \end{aligned}
        \right.\eqno(4.2)
    \end{equation*}
    So:
    \begin{equation*}
    \begin{aligned}
        W &= \frac{1}{2}\int D\cdot Edr = \frac{1}{2}\frac{Q}{(4\pi)^2}4\pi\{\frac{1}{\epsilon}\int_a^b\frac{1}{r^2}\frac{1}{r^2}r^2dr + \frac{1}{\epsilon_0}\int_b^\infty \frac{1}{r^2}dr\}\\
        &= \frac{Q^2}{8\pi^2}{\frac{1}{\epsilon}(-\frac{1}{r})|_a^b + \frac{1}{\epsilon_0}(-\frac{1}{r})|_b^\infty}\\
        &= \frac{Q^2}{8\pi \epsilon_0(1+\chi_e)}(\frac{1}{a}+\frac{\chi_e}{b})
    \end{aligned}
    \eqno(4.3)
    \end{equation*}

\end{document}