\documentclass[UTF8]{ctexart}
\usepackage{ctex}
\usepackage{amsmath}
\title{热力学与统计物理-第一周第一次作业}
\author{吴远清-2018300001031}
\begin{document}
	\maketitle
	1.一个粒子做随机行走,步长为l,走N步后,其距离出发点距离$r^2$与步数N关系如何,试推导。\\
	解:\\
	1).一维情况:\\
	记粒子沿X正向行走为$X=l$,沿X负向行走为$X=-l$,显然的,粒子的自由行走符合二项分布,且:
	$$P(X=l)\,=\,P(X=-l)\,=\,0.5\eqno(1.1)$$
	设在N次行走中,沿$X=l$发生了n次,则行走距离为:
	$$r\,=\,l\times n+(-l)\times (N-n)\,=\,l\times (2n-N)\eqno(1.2)$$
	此情况发生的概率为:
	$$P(n) = \frac{N!}{2^N n! (N-n)!}\eqno(1.3)$$
	则:
	$$\overline{r^2}\,=\,\sum_{n=0}^{N}l^2\times (2n-N)^2 \frac{N!}{2^N n! (N-n)!}\,=\,l^2\frac{N!}{2^N} \sum_{n=0}^{N}\frac{(2n-N)^2}{n!(N-n)!}\eqno(1.4)$$
	其中:
	$$\sum_{n=0}^{N}\frac{(2n-N)^2}{n!(N-n)!}\,=\,\frac{2^N}{(N-1)!}\eqno(1.5)$$
	将(1.5)式代入(1.4)式中:
	$$\overline{r^2}\,=\,N\times l^2\eqno(1.6)$$
	2).二维情况: \\
	2.1) On Lattice:\\
	与一维情况相似的,我们独立考虑X方向与Y方向的行走:\\
	假设在N次行走中,有$N_1$次发生在X方向上,有$N-N_1$次发生在Y方向上,由(1.6)式可知:
	\begin{equation*}
		\left\{
		\begin{aligned}
		&\overline{r_x^2}\,=\,N_1 \times l^2\\
		&\overline{r_y^2}\,=\,(N-N_1) \times l^2
		\end{aligned}
		\right.\eqno(1.7)
	\end{equation*}
	并且:
	$$\overline{r^2}\,=\,\overline{r_x^2}+\overline{r_y^2}\eqno(1.8)$$
	将(1.7)代入式(1.8)中得:
	$$\overline{r^2}\,=\,N \times l^2\eqno(1.9)$$	
	2).Off Lattice:\\
	对于每一次行走,总具有固定的长度l以及随机的方向$\theta,\, \theta$取为与X正向的夹角,对于N次行走,则有:
	$$\{\theta_1,\theta_2....\theta_N\}\eqno(1.10)$$
	分别有:
	\begin{equation*}
		\left\{
		\begin{aligned}
		&\overline{r_x^2}\,=\,l^2 \times (Cos^2\theta_1 + Cos^2\theta_2 + ... + Cos^2\theta_N) \,=\, l^2 \times \overline{Cos^2\theta}\\
		&\overline{r_y^2}\,=\,l^2 \times (Sin^2\theta_1 + Sin^2\theta_2 + ... + Sin^2\theta_N) \,=\, l^2 \times \overline{Sin^2\theta}
		\end{aligned}
		\right.\eqno(1.11)
	\end{equation*}
	由三角函数的周期性可得:
	\begin{equation*}
		\left\{
		\begin{aligned}
		&\overline{Cos^2\theta} \,=\, \int_{0}^{2\pi}Cos^2\theta d\theta \,=\, \frac{1}{2} + \frac{1}{2}\int_{0}^{2\pi}Cos(2\theta)d\theta \,=\, \frac{1}{2}\\
		&\overline{Sin^2\theta} \,=\, \int_{0}^{2\pi}Sin^2\theta d\theta \,=\, \frac{1}{2} - \frac{1}{2}\int_{0}^{2\pi}Cos(2\theta)d\theta \,=\, \frac{1}{2}
		\end{aligned}
		\right.\eqno(1.12)
	\end{equation*}
	即:
	\begin{equation*}
		\left\{
		\begin{aligned}
		&\overline{r_x^2} \,=\, \frac{N}{2}l^2\\
		&\overline{r_y^2} \,=\, \frac{N}{2}l^2
		\end{aligned}
		\right.\eqno(1.13)
	\end{equation*}
	最终得到:
	$$\overline{r^2}\,=\,\overline{r_x^2}+\overline{r_y^2} = N \times l^2$$
	3).三维情况:
	对于三维情况下,有无网格均与二维解法相似,不再重复证明(Off Lattice情况下取两个角来确定方向,即可相似的证明)
\end{document}
	